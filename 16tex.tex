\documentclass[12pt]{article}
\usepackage[utf8]{inputenc}
\usepackage[russian]{babel}
\usepackage{amsmath}
\usepackage{amssymb}
\usepackage{hyperref} % Включить ссылки в PDF
\begin{document}
    \tableofcontents % Вставить содержание
    \newpage
    \section{Определения и свойства}\\
    Пусть $q = p^m$, где $p$ --- простое. Рассмотрим множество $\mathbb{F}^n_q$, состоящее из~наборов 
    длины $n$ с~компонентами из~поля $GF(q)$. Расстоянием Хемминга
    $d({\bf x},{\bf y})$ междунаборами ${\bf x}$ и ${\bf y}$ называется число разрядов, в~которых эти
    наборы не совпадают. Весом $||{\bf x}||$ набора ${\bf x}$ из $\mathbb{F}^n_q$ называется число ненулевых разрядов этого набора. Нетрудно видеть, что
    \begin{equation}
        d({\bf x},{\bf y})=||{\bf x}-{\bf y}||.
        \label{1}
    \end{equation}
    Подмножество $G = \{{\bf g}_1,\dots, {\bf g}_m\}$ множества $\mathbb{F}^n_q$ называется {\it$q$-ичным кодом
    длины $n$ с кодовым расстоянием $d$,} если для любых двух его элементов ${\bf g}_i$ и ${\bf g}_j$ {\itрасстояние Хемминга междуними не меньше $d$.} Как и в двоичном случае, говорят, что код $G$ исправляет $t$ независимых ошибок, если его кодовое расстояние не меньше чем $2t+1$. Допустим, что под воздействием шума
    кодовый вектор ${\bf g}$ длины $n$ превратился в вектор ${\bf v}$ такой же длины. В этом случае ненулевые компоненты вектора ${\bf c=v-g}$ указывают положение ошибок, а их величины называются значениями ошибок. Используя равенство \eqref{1}, нетрудно показать, что для линейных $q$-ичных кодов справедливы теоремы, доказанные выше для двоичных кодов. В частности, имеют место следующие основные теоремы, доказательства которых полностью совпадают с доказательствами соответствующих теорем для двоичных кодов.
    
    {\bf Теорема 16.1}. {\it В каждом $q$-ичном линейном коде $G$ кодовое расстояние $d$ равно минимальному весу его ненулевого элемента:}
    \begin{equation*}
        d=\min_{{\bf g\not=0,g}\in G}||{\bf g}||.
    \end{equation*}
    
    {\bf Теорема 16.2.} {\it Для того, чтобы матрица ${\bf H}$ была проверочной матрицей 
    $q$-ичного линейного кода с кодовым расстоянием не меньшим $d$~необходимо и достаточно, чтобы любые $d-1$ столбцов матрицы ${\bf H}$~были линейно
    независимы.}
    
    Как и в двоичном случае, при помощи теоремы \eqref{2} можно легко определить 
    $q$-ичные коды, исправляющие однуошибку. Для того, что матрица ${\bf H}$
    с элементами из поля $GF(q)$ была проверочной матрицей кода, исправляющего
    однуошибку, необходимо и достаточно, чтобы в этой матрице не было
    кратных столбцов. Любая матрица, у которой в каждом столбце первая
    ненулевая компонента равна единице и все столбцы различны, удовлетворяет 
    этому условию. Например, матрица
    \begin{equation}
    \begin{pmatrix} 
        0&0&0&0&1&1&1&1&1&1&1&1&1\\ 
        0&1&1&1&0&0&0&1&1&1&2&2&2\\ 
        1&0&1&2&0&1&2&0&1&2&0&1&2
    \end{pmatrix}
    \label{2}
    \end{equation}
    будет проверочной матрицей троичного кода длины 13, исправляющего  однуошибку. Так как в $\{0, 1, 2\}^{13}$ шар радиуса один состоит из 27 наборов, а код с проверочной матрицей \eqref{2} состоит из $3^{10}$ элементов, то, очевидно, что этот код совершенный.

\end{document}